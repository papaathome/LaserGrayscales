\chapter{Introduction}\label{Introduction}

\GS\ is a tool to test the effects of your laser at power levels 1 through 255 and
at speeds ranging from 1 mm/s to 70 mm/s or 60 mm/min to 4200 mm/min and different
Z heights. This is a good way to see how your laser will react to different materials
at different speeds and power levels.

\GS\ was started `out of necessity' of having no proper test tools for a new laser
engraver tool. On the internet a \href{https://www.thingiverse.com/thing:3349071}{g-code script}
is available but its use of g-code is targeting different elements of the industry
standard RS274 format than what was acceptable for the g-code interpreter with my machine.
Changing the script manually was time consuming and just for one script where more testing
is required.

\GS\ works on the idea that you want to display the same image using different settings
for speed, laser power and Z height. You can plot an image `X' one time then change the settings
and plot it again (and again). \GS\ let you plot a line of images `X' `X'... `X' and specify
a different setting for each image. And it can do that for multiple lines. You also have
control over the amount white space between images and lines and between group of
images and group of lines. The idea of the grouping is that you more easily can identify which
setters are used

This will result in a grid with groups of images, e.g.: groups of items where
each item is a group of 2 by 2, with indexing numbers for each image X. (any configuration is possiblev)
\[
\begin{array}{llllllll}
    X_{1,m}   & X_{2,m}   &   & X_{3,m}   & X_{4,m}   & \dots  & X_{n\mhyphen 1,m}   & X_{n,m}\\
    X_{1,m\mhyphen 1} & X_{2,m\mhyphen 1} &   & X_{3,m\mhyphen 1} & X_{4,m\mhyphen 1} & \dots  & X_{n\mhyphen 1,m\mhyphen 1} & X_{n,m\mhyphen 1}\\
    \vdots    & \vdots    &   & \vdots    & \vdots    & \iddots & \vdots      & \vdots   \\
    X_{1,4}   & X_{2,4}   &   & X_{3,4}   & X_{4,4}   & \dots  & X_{n\mhyphen 1,4}   & X_{n,4}\\
    X_{1,3}   & X_{2,3}   &   & X_{3,3}   & X_{4,3}   & \dots  & X_{n\mhyphen 1,3}   & X_{n,3}\\
              &           &   &           &           &        &             &          \\
    X_{1,2}   & X_{2,2}   &   & X_{3,2}   & X_{4,2}   & \dots  & X_{n\mhyphen 1,2}   & X_{n,2}\\
    X_{1,1}   & X_{2,1}   &   & X_{3,1}   & X_{4,1}   & \dots  & X_{n\mhyphen 1,1}   & X_{n,1}\\
\end{array}
\]

The `image' X can be as simple as one straight line, a box (without any filling in it), a square
with an uniform grey filling or an image from file (yet to be implemented).

This allow you to make a crude first try to look for the sweet spot you need and then
to refine limiting the range and making smaller steps to find the best settings for your
application.

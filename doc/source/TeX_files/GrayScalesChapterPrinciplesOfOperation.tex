\chapter{Principles of operation}\label{PrinciplesOfOperation}

The program can be used to generate test patterns to find the sweet spot of operations for a particular image on some
material. It is also possible to create test patterns for testing the CNC hardware or the laser equipment up to a point.

To get the maximum out of \GS\ it helps to understand how it works. Not in terms of buttons and values but in how it was
designed and how information is used to produce test patterns.

The program applies a number op operations on an initial image\index{image}\index{image!initial}.
Each operation\index{operation} generates a new image. The last operation is producing G-Code for a test pattern.
Operations before that may apply a repeat pattern on the image or put a border around the image. It will be relatively
easy to implement a new operation at a later time.

For a first version of \GS\ a limited number of operations in a fixed order will be available. Giving the user more
flexibility in the order of applying operations is for a future version. The same for the initial images, a limited
number of initial images will be available. In future versions more types of images will become available.

There are a number of arguments that can be changed during the test, collectively called the
`{arguments under test}'\index{argument under test}.
When generating a test a original image\index{image!original} is repeated a number of times and each image will use
different and specific values for the arguments under test. Available for arguments under test are Z
height\index{argument under test!Z height}, speed\index{argument under test!speed} and
laser power\index{argument under test!power}.

\section{Design requirements}
A number of requirements were formed early for \GS.
\begin{itemize}
    \item Test patterns must always operate within target machine capabilities.
    \item Designing a test pattern must be possible for many G-Code consumers, not for just one machine or manufacturer.
    \item Produced G-Code must be minimal and adhere as much as possible to RS274-D.
    \item Generating G-Code must be  as fast as possible (in seconds, not in minutes).
    \item New users must be able to construct a test pattern safely without much concerns about how to use the program
          (low learning curve, result examples) but not hinder the experienced user (efficient user input).
    \item While configuring a test pattern it must give an indication where target machine capabilities are
          causing clipping of arguments under test on the image.
    \item While configuring a test pattern it must immediately indicate its size in x and y directions.
    \item While configuring a test pattern it must give a run time indication.
    \item The start point of a test pattern is an image to be realised with some arguments under test.
    \item Arguments under test include:
          \begin{itemize}
              \item Z height while engraving (for `out of focus' effects).
              \item Speed while engraving (for speed versus power effects).
              \item Intensity of laser while engraving (for speed versus power effects).
          \end{itemize}
    \item A number of operations must be available on the image including:
          \begin{itemize}
              \item Limits operator: All values used by the (final) image are within the capabilities range of the
                    target machine all the time.
              \item Pattern operator: The (final) image must be repeated first in X direction and then in Y direction.
              \item Border operator: The (final) image is given a border with some raster information.
              \item Rotation operator: The (final) image can be rotated  over any angle.
              \item Translation operator: The (final) image can be moved to any position
          \end{itemize}
    \item Operations must be repeatable e.g.: applying the Pattern operator twice will create an outer pattern over
          an inner pattern over an image.
\end{itemize}

\section{Images}
There is an image\index{image|textbf} to start with. This is the `original image' and the intension is to repeat it
several times and test it with different settings.

All images have is a width and an height. A number of images are available, more are in the planning.
Each image have some intended use. Each image have a specific purpose in mind. For each image some typical setting ranges
are given. These ranges are not mandatory but just sensible ranges for values to begin with. If the results are not good
do try some values outside the advised ranges.

\subsection{The line}
The purpose of the line\index{line|see {image}}\index{image!line} is to see the effects of some setting. It is the basis
of all testing.
On its own it does not give much information but chaining a number of lines in a row where all lines are the same but
for one the gradual effects of that one setting changing over the full length  will show the effect more clearly.
A next line above the first one  allows for a second setting to check the combined effect of two settings in an easy way.

Line lengths are defined in the x direction by setting the width. The typical setting falls in the range of 5 mm to 20 mm.
A line width is not defined. Note: The width is defined by the laser spot, very small and may be depending on how the laser
equipment is mounted if the spot is not round. In most cases it is a relative long and in the middle thin elliptic shaped.

Room between lines in the y direction are defined by setting the height. A typical setting is in de range of 0.25 mm to 0.5 mm.

\subsection{The box}
The purpose of the box\index{box|see {image}}\index{image!box} is to see the effect of laser power in the lower ranges
at higher speed.

A box is defined by four lines along its side. There is no filling in the box. A typical setting for width and height
is in the range of 50 to 100 mm. For higher speeds larger settings give better results.

While the G-Code interpreter is executing blocks of commands it has a few things to take care for. One is stopping at
(sharp) corners. When it is getting near a corner it must stop moving in one direction and start moving in another one.
This will take time and deviations from the selected speed.

Some G-Code interpreters will keep the laser at constant power all the time but others might be adaptive and when
decelerating for the corner also lower the laser power.

Lasers have the property that below some power\index{argument under test!power} threshold level they will not fire,
there is just not enough power to emit any light. Finding that level where the laser is guaranteed to fire is an
important step for getting good and consistent result.

Running a test with a box can produce tree results.
\begin{enumerate}
    \item The line ends at the corners are more black than in the middle.
    \item The line ends at the corner are equally black as in the middle.
    \item The line stops before it is at the corner.
\end{enumerate}

In the first case the conclusion is that the G-Code interpreter is not adaptive. Lower speeds result in more blackness.
Try to work with lower speeds in your project to reduce the effects of this.

In the second case the results are fine. The G-Code interpreter may or may not be adaptive. The settings for speed an
power are useful and give good results.

In the third case the G-Code interpreter is adaptive and is lowering the laser power below the threshold point of the
laser. Again, try to lower the speed to reduce the effects of this. Also, try to find out what the threshold level of
the laser is so you can keep that in mind when selecting a power level.

\subsection{The square}
The purpose of the square\index{square|see {image}}\index{image!square} is to see the effects of engraving a surface.

A square is defined as a box with a filling in lines per cm in it. Typical settings for width and height fall in the
range of 5 mm to 100 mm. Typical settings for lines/cm are 20 lines/cm to 50 lines/cm. (higher settings will increase
processing time proportionally.)

Typically when engraving a line it has a length and nearly any width. The laser is in focus and ideally all the laser
light is in one spot. This might result in an engraving with many visible lines, like an old copper engraving print.
This might or might not be the result to achieve. Raising Z up to a few mm will soften the lines. Lowering might give
the same results but with the effect that the laser equipment is getting nearer to the surface.

One way of softening this effect is getting the laser out of focus by raising the Z height\index{argument under test!Z height}
(and maybe raising the power to keep in check with the grey level) There is no measurable good or bad result for this
test, it all depends on the desired effects but this test give some more control over the end results.

\subsection{The card}
(Not yet implemented.)

The intention of the card\index{card|see {image}}\index{image!card} is to have a small representative part of the end
result and do a test run with it. There is no measurable good or bad result, it serves more as a check around the
selected settings for fine tuning or as a final check.

Once this test is available the it will also be possible to engrave the full image. Except that the software for
engraving at some speed and power settings might also have some operations on image manipulating such as sharpening,
softening, despeckle, gamma correction and more.

\section{Operations}
A limited set of operations is available. As a matter of fact only two and one is automatic applied. The other one
can be applied zero, once or two times. More operations and more freedom on applying or not is in development.

Each operation\index{operation}, except for the generator operation will result in a new image where the original
image is part once or more times. The generator operation creates a G-Code file.

\subsection{The Pattern}
The intention of the pattern%
\index{pattern|see {operation}}\index{operation!pattern}
is to repeat an image in x and y
direction with no or some space between all images.

Each image copy is numbered with grid coordinates, on the x axis running from (0, y), (1, y) ... (n, y) and similar
on the y axis from (x, 0), (x, 1) ... (x, m). The grit numbering is fixed and other operations will not change it.

Repeated pattern operations will result in multiple levels of coordinates and each original image\index{image!original}
will have a unique address. The unique address of an original image and the total count of original images on
the x axis (or y axis) is used to calculate values for arguments under test.

\subsection{The Grid and border}
(Not yet implemented.)

\subsection{The Rotation}
(Not yet implemented.)

The purpose of this operation is to rotate the image over a number of degrees at its centre.

\subsection{The Translation}
(Not yet implemented.)

\subsection{The G-Code generator}
Technically this is an operation\index{generator|see {operation}}\index{operation!generator} and it is applied as
the last operation. The results are not another image but a G-Code file.

This operation have several tasks.
\begin{itemize}
    \item Calculate values for arguments under test with each original image.
    \item Check and clip all values against target machine minimum and maximum values.
    \item Generate G-Code file suited to the target machine.
    \item Write G-Code to file.
\end{itemize}

\paragraph{Calculating values for arguments under test.}~\\
At this phase the maximum number of original images in x and y direction is calculated. Then for each
original image\index{image!original} the values for the arguments under test are calculated.

All images in the x or y direction may or may not be all in a straight line. X and y here are index numbers and not
positions on a XY plane. Rotating an image may result in direction changes but the numbering is not changed, e.g.:
rotating 90 degrees counter clockwise will result in the x numbering running in the y direction and the y numbering
in the -x direction. No corrections are made for this and values for arguments under test are applied as calculated
for the x and y value.

For each argument under test\index{argument under test} a value is calculated using the formula

\begin{equation}
    value_a = min_a + \frac{index_a}{index_{a, max}}(max_a - min_a)
\end{equation}

Where $a$ is $x$ or $y$, $min$ and $max$ are given limits by scale values and $index_a$ and $index_{a,max}$ depend on
position and number of original images in a given direction.

Then $value_x$ and $value_y$ are added together, this sum is used as the value for the argument under test for the
image with index $(x, y)$.

Example: Speed values is scaled in the X direction as $(0, 0)$ and in the Y direction as $(20, 50)$. Power values is
scaled in the X direction as $(100, 200)$ and in the Y direction as $(0, 0)$ and there are 10 original images in the
x direction and 4 in the y direction.

Then the speed (in mm/sec) and power value for image with index $(3, 2)$ is:
\[ speed_{x} = 0 + \frac{3}{10}(0 - 0) = 0 \]
\[ speed_{y} = 20 + \frac{2}{5}(50 - 20) = 32 \]
\[ speed = 32 \]
~\\
\[ power_{x} = 100 + \frac{3}{10}(200 - 100) = 160\]
\[ power_{y} = 0 + \frac{2}{5}(0 - 0) = 0\]
\[ power = 160 \]

One thing to notice is that it is possible to have scaled values for arguments under test in both X and
Y direction. The two will be added together and the results will be a skewed setting on the XY plane for that
argument under test.

\paragraph{Check and clip.}~\\
\index{clip}Before G-Code is generated all calculated values are checked against machine limits\index{machine limits}.
For configuring machine limits see the chapter on configuration details.

Each and every calculated value\index{calculated value} is checked against a minimum value and a maximum value.
If the calculated value is less than the minimum value then it is replaced by the minimum.
If the calculated value is greater then the maximum value it it replaced by that.

\Warning{It is not guaranteed that all values used are within machine limits\index{configuration!machinesettings}.
There are user supplied sections for intro, header and footer. The user can inject values\index{inject values} out of range.}

\paragraph{Generate G-Code.}~\\
The next phase is to generate\index{operation!generator|textbf} G-Code from the final image. The G-Code is partitioned in several sections.
\begin{itemize}
    \item \GS\ information details.
    \item User supplied info section (copied verbatim, no processing done)
    \item Test pattern details, all settings supplied in the user interface as an overview.
    \item X and Y ranges used by the test pattern.
    \item G-Code settings required by the script.
    \item User supplied header section (copied verbatim, no processing done)
    \item G-Code for the test pattern
    \item G-Code to safely stop the script
    \item User supplied footer section (copied verbatim, no processing done)
\end{itemize}

The three user supplied sections are copied verbatim. No checking or other forms of parsing is done on it.
This may be a cause of problems. It is possible to inject values that are out of machine acceptable limits
or a comment form is used that is not supported by the G-code interpreter.

\WarningCheckAndTest

\paragraph{Write G-Code.}~\\
Building and generating a G-Code file uses a build area\index{build area}. Its location is configurable and by default
`.\textbackslash{}build'. Intermediate results are stored here if required and so is the final result.

%The user interface also have the option to copy the final result to any other location.

\section{Arguments under test}
\index{argument under test|textbf}There are three values that can vary over a range of values.
The three being Z height\index{argument under test!Z height}\index{Z height|see {argument under test}},
speed\index{argument under test!speed}\index{speed|see {argument under test}} (for G1 codes)
and Laser power\index{argument under test!power}\index{power|see {argument under test}}.

Applied values range from a minimum value to a maximum value and are calculated linearly in X and Y direction
based on the number of original images\index{image!original} in each direction. The calculated values in X and Y
direction are added together and used as the final value.

On normal use a calculated value will have some scale in one direction and is constant in the other. Test patterns
using this configuration result in one argument changing from image to image in one direction and another one
changing in the other direction. Finding good or bad result is more or less checking a table.

It is possible for calculated values to have some scale in both directions. This can be useful if the normal use
will give skewed results. While skewing the scales in the other direction this can be compensated. Its use will
depend on specific test details.
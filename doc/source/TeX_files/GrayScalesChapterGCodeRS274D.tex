\chapter{G-Code and RS274-D}\label{GCodeAndRs274D}

G-Code and its use did not just happen `at once', Wikipedia have an informative page about \href{https://en.wikipedia.org/wiki/G-code}{G-code}
and the phrase \emph{`G-code has many variants'} in the first paragraph informs you that there is much to say
about it. The G-code that your machine is using might be a slightly different variation to the G-code that your
tool is generating (and \GS\ for that). In this chapter the general structure and some of the differences and pitfalls are discussed.

The RS274-D\index{RS274-D}\ document was approved in feb 1979 by the `Electronic Industries Association' (EIA)\index{EIA}.
Unfortunately a few subjects are mentioned but not defined or explained. Some subjects, such as a subroutine mechanism,
are not mentioned explicitly and industry is using a `post processing' phase to handle specific manufactured specifics.
By now the RS274-D has 40 years of history and still is the document referred to for G-Code compliance.

The following information about the RS274 is \emph{informative only} to give some additional insights in the use
of G-code and should not be used for design decisions. The ultimate source of information for such decisions is
the documentation with the equipment/software processing the actual G-Code.

\section{G-Code data structure}
G-code can be preceded with a line ending with a 'Start of Program'\index{Start of Program} or
'\%'\index{\%|see {Start of Program}} character.
This character can appear only once, at the end of the line. No lines preceding the line with 'Start of Program' are allowed.

G-code is defined as variable blocks of data with a `End of Block' (EOB) marker. Notably, the first block\index{block} should
be preceded by a EOB marker. Each block is a variable number of words. `End of Block' is a \grammarnl.

Words\index{word} are the actual G-code items you can recognise such as `G0', `X 123' and alike. The length (in characters) is
not fixed and there are several groups such as `Dimensional words', `Non-dimensional words' and subgroups.

The state machine handling blocks of date will, for words not sent in a block, substitute the value from the previous
block e.g: The state machine on processing block `G1 X1, Y2' will move to x=1 and y=2. On the next block `G1 Y3' the
state machine will substitute the value of 1 for x and move to x=1 and y=3.

The thing to remember is that a block of G-code words is one line of text e.g.: `G90\grammarnl' or
`G00~X0~Y0~F4200\grammarnl' are two blocks and each one is a set of words applied at the same time.

Also whitespace (spaces, tabs) have no meaning and nothing is said about lower case or upper case.
`g0\textvisiblespace{}x\textvisiblespace{}+0.0\textvisiblespace{}y\textvisiblespace{}+0.0\textvisiblespace{}f\textvisiblespace{}4200\grammarnl'
is identical to `G00X0000Y0000F4200\grammarnl'. Whitespace is used only for readability, numeric value specifications allow
for leading and trailing zero's (for dimensional values).

Comment\index{comment} text in G-Code is not explicit provided for in the RS274 except for a few odd remarks without
further explanations.

The character `)'\index{comment!)}\index{)|see {comment}} is defined as `Control In' and
`('\index{comment!(}\index{(|see {comment}} is defined as
`Control Out'. What the exact function is of these two is not explained but the general interpretation looks to be
`On Control In all following input characters is forwarded to the state machine handling blocks of data' and
`On Control Out all following input characters up to the next newline or On Control In character is ignored and not
forwarded to the state machine'. Effectively, any text between `(' and `)' (or the next \grammarnl) is ignored giving one
way of adding comments to a G-code file.

The character `\slash'\index{comment!\slash}\index{\slash|see {comment}} is defined as `Block Deleter (slash)' and
can appear as the first (non-white) character on a line. It is intended to quickly disable blocks without removing it. Effectively, any line starting with `\slash' is not processed giving another way of adding comment to a G-code file.

The character `!'\index{comment!$^^21$}\index{$^^21$|see {comment}} is used as `Start of Comment' by some C-Code interpreters. The RS274 does not
mention this character.

This depend on how the g-code interpreter is reading the file, test before you use it.

\section{Backus-Naur form}
This section can be skipped, it provides a more strict definition of G-code but does not introduce anything new to
the previous section. A more formal notation of the G-Code data in
\href{https://en.wikipedia.org/wiki/Backus%E2%80%93Naur_form}{Backus-Naur form} is given below. It does not represent
all the details in RS274 because it is composed without the use or Backus-Naur forms in mind making it hard to capture
some details.

All dimensional words (see below) should appear `in order'. That is not covered in the description below, other than
a remark, to keep the description reasonable short. Words of any type shall only appear once in a block. This is not
reflected in the grammar below, it will complicate the grammar considerable. Also the fact that missing words are
allowed in which case the state machine is substituting the value from the previous block is not represented  to keep
the grammar short and readable.
\begin{grammar}
    <gcode>      ::= <prog start> <EOB> <blocks>

    <prog start> ::= <empty> \textbar\ <any> <SOP>

    <blocks>     ::= <empty> \textbar\ <block> <blocks>

    <block> ::= <N word>\index{block}\\
                     <G words>\\
                     <dimension words>\\
                     <interpolation parameter words>\\
                     <F word>\\
                     <S word>\\
                     <T word>\\
                     <M words>\\
                     <EOB>

    <N word>     ::= `N' INTEGER\index{word}

    <G words>    ::= <empty> \textbar\ `G' <word part> <G words>

    <dimension word> ::= <dimension address> <value> <optional F word>

    <dimension address> ::= ~\\`X'\textbar `Y'\textbar `Z'\textbar `U'\textbar `V'\textbar `W'\textbar `P'\textbar `Q'\textbar `R'\textbar `A'\textbar `B'\textbar `C'\textbar `D'\textbar `E' \\// appearing in this order.

    <interpolation parameter words> ::= <interpolation address> <word part>

    <S word> ::= `S' <word part>

    <T word> ::= `T' <word part> <optional D word>

    <F word>     ::= `F' <value>

    <M words>    ::= <empty> \textbar\ `M' <word part> <M words>


    <value>      ::= REAL \textbar\ INTEGER

    <word part>  ::= <sequence number> <preparatory function number>


    <any>    ::= [\^{}\%\textbackslash n]*

    <SOP>    ::= `\%'

    <EOB>    ::= `\textbackslash n'
\end{grammar}

